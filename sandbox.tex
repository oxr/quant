

And the first line of attack is to try
to lift the powerset functor from $\Set$ to $\Rel$. That would allow
us to take constructions in $\Rel$ to constructions in $\Rel$ which
involve powersets. More exactly, we'd like to have a functor that acts
as $2^(-)$ on objects. The question is what should be the action on
arrows.  One possibility is to use the action of $2^{(-)}$ on
functions because relations are functions of the form $X \to 2^Y$ and
applying $2^{(-)}$ to it gives a function of the same form namely $2^X
\to 2^{2^Y}$. However, this definition is not very practical as it
defines a relation between $2^X$ and $2^Y$ which relates each $X'
\subseteq X$ to a $Y' \subseteq Y$ iff $Y'$ is the image of some $x
\in X'$. We need something else.  Here is another possibility:
define the powerset relation of a relation $R : X \to Y$ to be:
%
\begin{equation}\label{eq:power}
X' ~ \sim_{\exists R} ~ Y' \quad \equiv \quad \forall y \exists x . x \sim_R y
\end{equation}
% 
The trouble is this doesn't define a functor as the following
calculation shows, using $x \sim_\id y
\equiv x = y$. Then 
\begin{align*}
    X' \sim_{\exists \id} Y' & ~ \equiv ~ \forall y \in Y' ~ \exists x \,.\, x \sim_\id y\\
&~\equiv~ \forall y \in Y' ~ \exists x\,.\, x = y \\
&~\equiv~\forall y \in Y' \,.\, y \in X'\\
& ~\equiv ~ Y' \subseteq X'
  \end{align*}
%
So $\exists \id$ is the down-closure of the identitiy. This is a general
fact for all relations of the form $\exists R$, namely that whenever $X
\sim_{\exists R} Y$ then also $X' \sim_{\exists R} Y'$ for $X \subseteq X'$ and
$Y' \subseteq Y$. \marginpar{\tiny Something to say about\\weakest
  preconditions\\and strongest\\postconditions? }

\begin{remark}
It follows that a subset of $X' \subseteq X$, i.e. a relation $X' : 1 \to X$, is
mapped to $2 \to 2^X$, which relates $1$ to the principal ideal of
$X'$, i.e. the set of subsets of $X'$, and $0$ to nothing. On the
other hand, and dually, $X' : X \to 1$ is mapped to $2^X \to 2$ such
that all of $X'$ and all its superset are related to $1$ and nothing
to $0$. 
\end{remark}

This could be useful though. Luckily, $\exists {(-)}$ preserves composition
as the following calculation shows:
\begin{align}
X' \sim_{\exists R};\sim_{\exists S} Z'& ~ \equiv ~\exists Y'. Y' \sim_{\exists S} Z' \wedge X' \sim_{\exists R} Y'\notag\\
\label{eq:comp1}  & ~ \equiv ~ \exists Y'.(\forall z\in Z'\exists y\in
Y'. y \sim_S z)\wedge (\forall y\in Y' \exists x \in X'. x \sim_R y) 
\end{align}

%
This should be equivalent to:
\begin{align}
X' \sim_{\exists {R;S}} Z' &~\equiv~ \forall z\in Z' \exists x \in X'. x~ R;S~ z\notag\\
\label{eq:comp2}&~\equiv~ \forall z\in Z' \exists x\in X'\exists y\in Y. x \sim_R y \wedge y \sim_S z
\end{align}
To show that \eqref{eq:comp1} implies \eqref{eq:comp2}, assume a $z \in
Z'$, by the first component of \eqref{eq:comp1} we have a $Y'
\subseteq Y$ and a $y \in Y'$
such that $y \sim_S z$. By the second component of \eqref{eq:comp1},
we have for this $y$ and $x \in X'$ such that $x \sim_R y$, as requred
by \eqref{eq:comp2}. On the other hand, assume \eqref{eq:comp2} and
define $Y'$ to be the set of all $y \in Y$ that are given by the
existential in \eqref{eq:comp2} for some $z$. Then clearly for all $y
\in Y'$, the conclusion of \eqref{eq:comp1} holds. 

So \eqref{eq:power} doesn't define a functor $\Rel \to \Rel$ but
nevertheless it preserves commuting diagrams. 

Next we define frobenius algebras. This is easy as we just use the
usual definition for sets and relations. The following observation
might be useful: 
\begin{remark}
The relation $\Delta_{\exists X} : 2^X
\to 2^X \x 2^X$ is equal to $2^X \to^{\exists \Delta} 2^{X \x X}
\to^{[\Delta,\Delta]_\ast} 2^{X + X} \cong 2^X \x 2^X$.  $2^\Delta$ relates
a subset $X' \subseteq X$ to subsets $\{ (x,x) | x \in X' \}\subseteq
X \x X$. The inverse image ${[\Delta,\Delta]_\ast}$ unzips the diagonal
to a pair of identical subsets of $X$. Thus, the composition is equal
to $\Delta$.
\end{remark}


% It is well known \cite{beck69,mulry94} that every lifting of a monad, $\mathbf{T}$, to the
% Klesli category of another monad, $\mathbf{S}$, is given by a natural
% transformation $\lambda: \mathbf{TS} \dot{\rightarrow} \mathbf{ST}$
% satisfying \emph{(certain axioms)}, where by a lifting of $\mathbf{T}
% : \cat{C} \to \cat{C}$ to $\cat{C}_{\mathbf{S}}$ we mean a functor
% $\mathbf{T}_{\mathbf{S}} : \cat{C}_{\mathbf{S}} \to
% \cat{C}_{\mathbf{S}}$ such that $\mathbf{F}\cdot \mathbf{T} =
% \mathbf{T}_{\mathbf{S}}\cdot \mathbf{F}$. 

% In our case, $\mathbf{T} = \mathbf{S} = \wp{-}$, the powerset monad, and
% $\lambda_X: \wp{\wp{ X}} \to \wp{\wp{X}}$ takes $x \in \wp{\wp{X}}$ to the
% principal ideal\footnote{The principal ideal on a set $X$ is a subset
%   of $\wp{X}$ closed under subsets and finite unions.} on $\bigcup
% x$. 

% In elementary terms, $\lambda$ takes a set of subsets $X'$ of $X$ to the
% set of \emph{all} subsets of the union of $X'$. 
% In terms of relations, the lifting has the following interpretation:
% it takes a relation $r : X  \to Y$ to the relation $\wp{r} : \wp{X}
% \to \wp{Y}$ for which $X' \subseteq X \sim_{\wp{r}} Y' \subseteq Y$ iff
% $\forall y \in Y'. \exists  x \in X'. x \sim_r y$. 

\section{$\exists R$, $\forall R$ and Comparison with adjoints}
In words, $\exists R : 2^X \to 2^Y$ holds between $X' \subseteq X$ and
$Y' \subseteq Y$, iff for each $y \in Y'$ $X'$ contains some element
related by $R$ to $y$. Formally,
\begin{equation}\label{eq:existsR}
X' \sim_{\exists R} Y' \quad \equiv \quad \forall y \in
Y'\,\exists x \in X'.\, x \sim_R y
\end{equation}
% 
The other quantifier therefore is one where $X'$ contains with yeach $y\in
Y$ \emph{all} elements of $X$ related to $y$. Formally:
\begin{equation}\label{eq:forallR}
X' \sim_{\forall R} Y' \quad \equiv \quad \forall y \in Y' \, \forall
x \in X.\,x \sim_R y \rightarrow x \in X'
\end{equation}
% 
Note that $y$ occurs negatively in the above definitions while $x$
occurs positively. It follows that 
\[
X' \sim_{\{\forall,\exists\}R} Y' \quad \implies \quad X''
\sim_{\{\forall,\exists\}R} Y'' \quad \text{for }X' \subseteq X''
\text{ and }Y'' \subseteq Y'
\]

\medskip
In $\Set$, a function $f : X \to Y$ implies a \emph{pullback
  function} on powersets $f_* : 2^Y \to 2^X$ defined by
precomposition:
%
\begin{equation}\label{eq:functorial-relation}
x \in f_* Y' \equiv f(x) \in Y'
\end{equation}
%
When $f$ is a relation, this is no longer clearly defined as there may
be many images of $x$ under $f$, some of which are in $Y'$ and some of
which aren't. But as long as $f$ is a relation which is a function,
i.e. it satisfies
\[ \forall x\,\exists! y.\, x \sim_f y \]
one can define the pullback of $f$ and talk about the left and right
adjoins to it. 
The pullback relation of function will be denoted $* f$. 
\begin{lemma}
  Let $f : X \to Y$ be a function in $\Rel$ (a relation which
  satisfies \eqref{eq:functorial-relation}), then $\exists f \dashv *f$ 
\end{lemma}
\begin{proof}
  Relations form a bicategory where 2-cells are the ordering of
  relations thus we want to show that $\exists f: 2^X \to 2^Y$ and
  $*f: 2^Y \to 2^X$ are adjoint in the bicategorical sense. It means
  we must how that 
\[
\id \subseteq \exists f ; * f\qquad *f ; \exists f \subseteq \id
\]
To prove the left-hand inequality, we need 
\[ 
\forall X' \subseteq X . X' \sim_{\exists f ; * f} X'
\]
which is that there exists a $Y' \subseteq Y$ s.t. $X' \sim_{\exists
  f} Y'$ and $Y' \sim_{* f} X'$; i.e. a $Y'$ such that
\begin{align*}
\forall y \in Y'. \exists x \in X'. x \sim_f y\\
&\text{ and }
\end{align*}
\end{proof}



\section{Ideals and Relations}
\ondrej{Roughly: properties are relations $\wp{X} \to \wp{Y}$ which
  are downclosed -- if a collections of things has the property, each
  it's subset has the property, closed under unions -- if things in two
  collections have the property, then each thing in their union has
  the property, contains the empty set -- trivially. This is to be
  worked out.}


% The functor
% $\mathbf{F}\mathbf{U}$, where $\mathbf{F} \dashv \mathbf{U}$ is the
% splitting of $\wp-$ fits the type but not the purpose as it takes a
% set $X$ to the functional relation whose only result is the union of
% $R$ applied to each 
% . It takes 
% a relation $R : X \to Y$ to the relation $2^R = \eta\mu: 2^X \to 2^Y$

% , we use the
% left adjoint $F: \Set \to \Set_{\wp-}$.


% lift
% $\wp-$ to $\Rel$ using a \emph{distributive law} \cite{Beck69}. Define
% $\wp- : \Rel \to \Rel$ on objects by $X \mapsto \wp X$, and on arrows as
% follows: a relation $R : X \to Y$ is mapped to that relation $\wp R :
% \wp X \to \wp Y$ which relates every subset of $X$ to all subsets of
% $Y$ which arise from $X$ by replacement of elements under $R$. In
% other words, when relations are understood as nondeterministic partial
% functions $\wp R$ captures all possible nondeterminism on collections
% of elements. More formally, identify $\Rel$ with the Kleisli category
% of $\wp-$. There is a distributive law, a natural transformation
% $\lambda_X : \wp{\wp X} \to \wp{\wp X}$, which takes a set of subsets
% of $X$ into the set of its crossections.

% For example: 
% \begin{align*}
% \lambda (\{\{a,b\},\{c\}\}) & \quad = \quad \{\{a,c\},\{b,c\}\}\\
% \lambda (\{\{a\}\},\{\{b\}\}) & \quad = \quad \{ a , b \} \\
% \lambda (\{\{a,b\},\{c\},\emptyset\}) &\quad = \quad \emptyset
% \end{align*}

% It follows from the theory in \cite{Beck69} that $\wp-$ lifts to $\Rel$ where a
% relation $R :  X \to \wp Y$ is taken to $\wp R : \wp X
% \xrightarrow{\wp r} \wp {\wp Y}
% \xrightarrow{\lambda_Y} \wp {\wp Y}$.


% {\bf Example:} The Frobenius relation $\delta$ is a map : $x \mapsto
% (x,x)$. It follows that \[2^\delta = X \mapsto \{ (x,x) | x \in X
% \}\].  

% \begin{quote}
% {\bf O.R.} Now it all works out :)
% \end{quote}


% Examples of calculations with determiners: exists, all, some, two





%For this part, we work in the category $Rel$ of sets and relations. This is compact closed as follows \cite{BobEric}. 
%
%Take $N$ to be a vector space with a fixed basis ${\cal B}$, where ${\cal B}$ is a finite set. A  basis vector of $N$  is denoted by $\ov{n}_i$. Given a set of individuals ${\cal U}$,  each  individual is mapped to a basis vector of $N$, the map is denoted by $\pi \colon {\cal U} \to {\cal B}$.  Thus a subset of basis vectors of $N$ represents a subset of individuals.  For example,    the sum $\sum_i \ov{n}_i$ denotes the `men' subset of individuals, where $i$ ranges over the basis vectors that are mappings of the individuals that are men. 
%
%Take $S$ to be the one dimensional space free over the singleton $\ov{1}$. The  zero vector represents false, and any nonzero value represents a degree of truth. 
%
%A transitive verb $w$, which is a vector in the space $N \otimes S
%\otimes N$, is represented by 
%\[
%\ov{w} := \sum_{ij} \ov{\{{n}\}}_i \otimes \ov{1}  \otimes \ov{\{n\}}_j, \quad 
%\text{if} \  \pi^{-1}({n}_i) \ w\mbox{'s} \  \pi^{-1}({n}_j)
%\]
% For example, the verb ``stroke'', denoted by $\ov{stroke}$, is represented by $
%\sum_{(i,j) \in  R_{stroke}} \ov{\{n\}}_i \otimes  \ov{1}
%\otimes \ov{\{n\}}_j$, for  $R_{stroke}$  the set of all pairs $(i,j)$ such that $\pi^{-1}(\ov{n}_i)$ strokes $\pi^{-1}(\ov{n}_j)$.  An intransitive verb ``sneeze'', denoted by $\ov{sneeze}$, is represented by $\sum_{i \in R_{sleep}} \ov{\{n\}}_i \otimes \ov{1}$, such that $\pi^{-1}(\ov{n}_i)$ sneezes.  
%
%
%The meaning of the sentence ``Det Sbj Verb''is a vector obtained by computing the following, which corresponds to the categorical morphism of the normalised diagram of the sentence.  
%
%\[
%(\epsilon_N \otimes 1_S) \circ (Det \otimes  \mu_N \otimes 1_S) \circ (\delta_N \otimes 1_{N \otimes S})\Big(\ov{Sbj} \otimes \ov{Verb}\Big)
%\]
%This is computed in three steps. First we compute the following
%
%\begin{align*}
%(\delta_N \otimes 1_{N \otimes S})\Big(\ov{Sbj} \otimes \ov{Verb}\Big) = 
%(\delta_N \otimes 1_{N \otimes S})\Big((\sum_i \ov{n}_i) \otimes (\sum_j \ov{n}_j  \otimes \ov{1}) \Big) =\\
%\delta_N(\sum_i \ov{n}_i) \otimes (\sum_j \ov{n}_j \otimes \ov{1}) = (\sum_i \ov{n}_i \otimes \ov{n}_i) \otimes (\sum_j \ov{n}_j \otimes \ov{1})
%\end{align*}
%
%\noindent
%Then we proceed by
%
%\begin{align*}
%(Det \otimes  \mu_N \otimes 1_S)\Big(\sum_i \ov{n}_i \otimes \ov{n}_i) \otimes (\sum_j \ov{n}_j \otimes \ov{1}\Big) = \\
%Det(\sum_i \ov{n}_i) \otimes  \mu_N(\sum_i \sum_j \ov{n}_i \otimes \ov{n}_j) \otimes 1_S(\ov{1}) = \\
%Det(\sum_i \ov{n}_i) \otimes  (\sum_i \sigma_{ij} \ov{n}_i) \otimes \ov{1} 
%\end{align*}
%
%\noindent
%The final step is as follows:
%
%\begin{align*}
%(\epsilon_{N} \otimes 1_S)  \Big(Det(\sum_i \ov{n}_i) \otimes (\sum_i \sigma_{ij} \ov{n}_i) \otimes \ov{1} \Big) =   \langle \ov{w}_k \mid \ov{\{\sum_i \ov{n}_i\}} \rangle  \otimes \ov{1}
%\end{align*}
%
%\noindent 
%where $Det (\sum_i \ov{n}_i) = \sum_k \ov{w}_k$, for $w_k \subseteq B_N$ where $B_N$ is the set of basis vectors of $N$.  Meanings of sentences with quantified objects and transitive verbs are computed in an identical fashion. 
%
%
%
%
%As an example suppose we have  a set of men $\{m_1, m_2\}$ and a set of cats $\{c_1, c_2\}$. Now suppose that one of the men $m_1$ and all of the cats  sleep, that is we have
%\[
%\ov{sleep} \ = \  \ov{m}_1 \otimes \ov{1} + \ov{c}_1 \otimes \ov{1} + \ov{c}_2 \otimes \ov{1}
%\]
%Here are some sample sentences:
%\begin{enumerate}
%\item The meaning  of the sentence `some men sleep', is computed in three steps. In the first step, we obtain
%\[
%(\ov{m}_1 \otimes \ov{m}_1 + \ov{m}_2 \otimes \ov{m}_2)  \otimes ( \ov{m}_1 \otimes \ov{1} + \ov{c}_1 \otimes \ov{1} + \ov{c}_2 \otimes \ov{1})
%\]
%In the second step we obtain
%\[
%Some(\ov{m}_1 + \ov{m}_2) \otimes {\cal P}(\ov{m}_1) \otimes \ov{1} = 
%(\ov{\{\ov{m}_1\}} + \ov{\{\ov{m}_2\}} + \ov{\{\ov{m}_1, \ov{m}_2\}}) \otimes (\ov{\{\ov{m}_1\}}) \otimes \ov{1}
% \]
%In the last step we obtain $\ov{1}$, since we have $\left \langle \ov{\{\ov{m}_1\}} \mid \ov{\{\ov{m}_1\}}) \right \rangle = 1$. This  means that the meaning of this sentence is true.  
%\item The meaning of `all men sleep' would be false. Since, in this case, in the second step we would obtain
%\[
%All(\ov{m}_1 + \ov{m}_2) \otimes {\cal P}(\ov{m}_1) \otimes \ov{1} = 
%(\ov{\{\ov{m}_1, \ov{m}_2\}}) \otimes (\ov{\{\ov{m}_1\}}) \otimes \ov{1}
% \]
%which would result to a 0 in the third step, since $\left \langle \ov{\{\ov{m}_1, \ov{m}_2\}} \mid \ov{\{\ov{m}_1\}}) \right \rangle = 0$.  
%\item The sentence `half of  men sleep' would also be true, since in the second step we obtain
%\[
%Half(\ov{m}_1 + \ov{m}_2) \otimes {\cal P}(\ov{m}_1) \otimes \ov{1} = 
%(\ov{\{\ov{m}_1\}} + \ov{\{\ov{m}_2\}}) \otimes (\ov{\{\ov{m}_1\}}) \otimes \ov{1}
% \]
%since we  have $\left \langle \ov{\{\ov{m}_1\}} \mid \ov{\{\ov{m}_1\}}) \right \rangle = 1$, in the third step we will obtain a $\ov{1}$.  
%\item The sentence `no man sleeps' will be false, since in the second step we obtain
%\[
%No(\ov{m}_1 + \ov{m}_2) \otimes {\cal P}(\ov{m}_1) \otimes \ov{1} = 
%(\ov{\{ \ \}}) \otimes (\ov{\{\ov{m}_1\}}) \otimes \ov{1}
% \]
% This, in the third step,  will results in $\left \langle \ov{\{\ \}} \mid \ov{\{\ov{m}_1\}}) \right \rangle = 0$.
%\end{enumerate}

