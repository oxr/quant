\section{Quantification in Sets}
%%% Local Variables: 
%%% mode: latex
%%% TeX-master: "Quant"
%%% End: 


The category $\Rel$ of sets and relations is a basic example of a
symmetric monoidal compact closed category. The tensor is Cartesian
product, $\times$, and monoidal the unit is the one element set,
$1$, $A^\ast = A$. 
Monoidal closure is therefore just the product. 
As an example consider the correspondence between relations and subsets of
the Cartesian product expressed formally as
\[
\Rel(1, X \x Y) ~\cong~\Rel(X,Y)~\cong~\Rel(X\x Y , 1)\]
%
where we are repeatedly using compact closure and the fact that
relation beween $1$ and $Z$ are demonstrably subsets of $Z$. 

\ondrej{This is probably best put where compact
  closed cats are introduced.}

The frobenius maps $\Delta$ is just 
the diagonal relation $x \sim (x,x)$ and $\nabla$ its transpose; $\bot : 1
\to X$ is the so-called \emph{fan} relation: $\ast \sim x, ~  \forall
x$, $\top$ its converse. The required axioms are easilly checked. 
\ondrej{This is probably best put where Frobenius structure is introduced.}


In order to introduce quantification in $\Rel$, we first discuss the
well known case in $\Set$ where the existential and universal
quantifiers are left and right adjoints, respectively, to the inverse
image functor.
%
More formally this can be described as follows. For a set $X$, the set
of functions from $X$ to the two-element set, $2^X$, is the powerset
of $X$.  For a function $f : X \to Y$, precomposition with $f$ is
internally the map $2^f : 2^Y \to 2^X$ that takes a $h$ to $hf$.  It
has both left and right adjoints with respect to the pointwise subset
ordering. In more detail, the powerset of $X$ ordered by subset
inclusion is isomorphic to the set of characteristic functions on $X$,
$2^X$, ordered pointwise by the ordering $0 < 1$ of $2$.  Any
pre-ordered set can be seen as a category\footnote{internal
  category?}. The function $2^f$ becomes a functor\footnote{internal functor?}
from the category $2^Y$ to $2^X$. This functor has both adjoints. It
is well known that when $f$ is the first projection $\pi_1 : X\x Y \to
X$, $2^{\pi_1}$ corresponds to \emph{weakening} in the sense that it
takes predicates over context $X$ to predicates over the larger
context, $X \x Y$, where the second component in $Y$ doesn't occur.
The left adjoint to $2^{\pi_1}$ interprets existential quantification,
and the right adjoint interprets universal quantification. In more
detail, the left adjoint, $\exists_f : 2^X \to 2^Y$, is defined by
%
\begin{equation}\label{eq:di}
\exists_f (X' \subseteq X) = \{ y \in Y ~|~ \exists x \in X. f x = y
\wedge x \in X'\} 
\end{equation}
%
i.e. the image of $X'$ under $f$. It is also known as the
\emph{direct image} of $f$.
The right adjoint $\forall_f : 2^X
\to 2^Y$ is defined by 
%
\[
\forall_f (X' \subseteq X) = \{ y \in Y ~|~ \forall x \in X. f x = y
\implies x \in X' \} \] 
%


There are other functions from $2^{X\x Y}$ to $2^X$,
such as the one sending an $f$ to the $g$ for which
$g(x) = 1$ if and only if $f(x,y) = 1$ for exactly two distinct values
of $y$. This quantifier could be called ``two''. Another example is
``none'' for which $g(x)=1$ iff $f(x,y) = 0$ always. This exhibits the 
existential and universal quantifiers as two extreme cases of a
spectrum of all possible quantifiers\footnote{Uhm, yes. That is
  precisely what it means to be a left and right adjoint.}  .  

In the following we assume for simplicity that the ambient context,
$X$, is empty, i.e. $X = 1$, and then quantifiers become certain
functions $2^Y \to 2$. Everything we say generalises
straightforwardly for arbitrary $X$.  In this case, the existential quantifier
becomes the characteristic function of nonempty subsets of $Y$; the
universal quantifier is the characteristic function of the singleton
$\{Y\}$, and ``two'' is the characteristic function of two-element sets.

In single sorted first order predicate logic, quantification ranges
over the whole domain $Y$, which is simply assumed to be fixed, and
often implicit, throught the formula. For example, in $\exists x. x >
0$, the bound variable, $x$, ranges over some ordered set. In
linguistics, however, the range of quantification is explicitly
stated, as in ``all men sleep'', where the word ``men'' restricts the
range of the quantifier, ``all'', to just men. In other words, the
quantifier is ``all men'', and ``all'' is a particle which expects a
linguistic category to become a quantifier. Formally, when we fix a
domain of all possible subjects of quantification, $Y$, such as all
nouns, a quantifier $Q$ must have type
\begin{equation}\label{eq:q}
Q ~: ~2^Y \to 2^{2^Y}
\end{equation}

Here, the first argument to $Q$ is the range of the quantification. The
result is a quantifier which possibly ignores everything that is
outside the range.

For instance, ``some'' ($\exists$) takes a subset
$\;\mathrm{men}\subseteq X\;$ to all nonempty subsets of $\mathrm{men}$; all
($\forall$) maps $\;\mathrm{men}\subseteq X\;$ to $ \{ \mathrm{men}
\}$; ``two'' takes $~\mathrm{men} \subseteq X$~ to two-element subsets of
$\mathrm{men}$.

\section{Relations}
We show two alternative categorical views on relations. The first one
understands a relation $R$ between sets $X$ and $Y$ as a function into
a powerset of $Y$, namely the function $r$
that assigns to each $x$ the set of all $y\in  Y$
related to $x$. Formally, $r(x) = \{~y \in Y ~|~ x \sim_R Y~\}$. 
On the other hand the same relation $R$ is a subset of the Cartesian
product $X \x Y$ of those pairs $(x,y)$ for which $x \sim_R y$. We now
give the standard Categorical presentation of these two points of view on
relations. 

\subsection*{Relations as Kleisli arrows}
The assignment $X \mapsto 2^X$ extends to a functor $2^{(-)} : \Set \to
\Set$ where the action on arrows is direct image \eqref{eq:di}.
The functor is a monad where the unit, $\eta$, maps $x$ to $\{x\}$ and
multiplication, $\mu$, is set union. Recall \cite{CWM} that for any
monad $(\mathbf{T}, \mu, \eta)$ on $\mathbb{C}$, the Kleisli category
of $\mathbf{T}$, denoted $\mathbb{C}_{\mathbf{T}}$, has the same
objects as $\mathbb{C}$. Arrows $X \to Y$ in $\mathbb{C}_{\mathbf{T}}$
are arrows $X \to \mathbf{T}Y$ in $\mathbb{C}$. Moreover, there is an
adjunction $\mathbf{F} \dashv \mathbf{U} : \mathbb{C} \to
\mathbb{C}_\mathbf{T}$ such that $\mathbf{F}$ is identity on objects,
and $f \mapsto \eta \cdot f$ on arrows and $\mathbf{U}$ is $\mathbf{T}$ on
objects and $\mu\cdot\mathbf{T}f$ on arrows. Moreover,
$\mathbf{U}\mathbf{F} = \mathbf{T}$.

Thus by $\Rel$ one can understand the Kleisli category of $2^{(-)}$, which is 
understood as the category of sets and relations where a function $r :
X \to 2^Y$ defines a relation $\sim_R$ by
\[
x ~\sim_R~ y \quad \equiv \quad y \in fx
\]


\subsection*{Relations as spans in $\Set$}
On the other hand the same relation $R$ defines a subset of $X \x Y$
by $\{~(x,y) ~|~ x \sim_R y~\}$. Such a set is called the
\emph{tabulation of $R$}. In $\Set$, such a subset is a \emph{span}
over $X$ and $Y$:
\[\bfig
\Atriangle/->`->`{}/[R`X`Y;p`q`]
\morphism/<-/[X`X\x Y;\pi_1]
\morphism(500,0)[X\x Y`Y;\pi_2]
\morphism(500,500)|m|/-->/<0,-500>[R`X\x Y;\langle p,q \rangle]
\efig\]
Such spans form a (bi)category with unit given by:
\[\bfig
\Atriangle/->`->`/[X`X`X;\id`\id`]
\efig\]
And composition using the pullback as follows:
\[\bfig
\Atriangle/->`->`{}/[R`X`Y;p`q`]
\Atriangle(1000,0) /->`->`{}/[S`Y`Z;r`s`]
\Atriangle(500,500) /->`->`{}/[R\x_\bullet S`R`S;`{}`]
\efig\]
where $R\x_\bullet S$ is the pullback of $q$ and $r$.

\newcommand{\Span}{\mathrm{Span}}
For an arbitrary category $\cC$, the category of spans in $\cC$,
formally $\mathrm{Span}(\cC)$, is a bicategory where a 2-cell $R
\Rightarrow S : X \leftrightarrow Y$ is given
by the vertical arrow in the following diagram:
\[\bfig
\Atriangle/->`->`{}/[R`X`Y;p`q`]
\Vtriangle(0,-500)/`<-`<-/[X`Y`S;`r`s]
\morphism(500,500)/-->/<0,-1000>[R`S;]
\efig\]
%
We leave out the details that this indeed forms a bicategory. We leave
the compact closed monoidal structure of $\Span(\Set)$ as an easy
exercise. 

Finally, note that the
correspondence between spans and relations is an equivalence rather
than an isomorphism, as there may be many spans representing
(tabulating) the
same relation. For instance, whenever $X \to/<-/^p R \to^q Y$ tabulates a
relation, $X \to/<-/^{[p,p]} R + R \to^{[q,q]} Y$ tabulates the same relation.
However, all such spans are equivalent via suitable
2-cells\footnote{Not true! $R+R$ is not $\cong R$}. Thus, strictly speaking $\Rel$ is the quotient
posetal\footnote{where between each pair $f,g : X \to Y$ of 1-cells is
  at most one 2-cell.}
bicategory $\Span(\Set)/_\simeq$ where
$\simeq$ is the above-described equivalence.

%
\subsection{Relations in Compact Closed Categories}
The construction of the category of relations $\Span(\cC)$ can be
generalised to an arbitrary category. The resulting category is
compact closed if $\cC$ is. Here are some of the details. 


\section{Powerset Relations}
In this section we study relations over powersets of sets,
i.e. relations between sets $2^X$ and $2^Y$ for some sets $X$, $Y$.

\subsection{Definition}
The following is the key observation of this section: 
\begin{proposition}
  A relation between $2^X$ and $2^Y$ is equivalently a set $T$
  together with a pair of \emph{relations} $p : T \to X$ and $q : T
  \to Y$, i.e. a span of relations, i.e. a relation in
  $\Rel$. 
\end{proposition}
\begin{proof}
Given a relation $R$ define $T = \{ ~ x \in 2^X , y \in 2^Y ~|~
x\sim_R  y\}$. Such $T$ comes with two projections $\pi_X : T \to 2^X$
and $\pi_y : T \to 2^Y$, which are relations in the first sense. In a
diagram, in $\Rel$ (!):
\[\bfig
\Atriangle/->`->`{}/[\{ ~ x \in 2^X , y \in 2^Y ~|~
x\sim_R  y\}`X`Y;\pi_X`\pi_2`]
\efig
\]
%
On the other hand given a span of relations:
\[\bfig
\Atriangle/->`->`{}/[R`X`Y;\pi_X`\pi_2`]
\efig
\]
define a relation between $2^X$ and $2^Y$ as in the general case by
\[
x \subseteq X ~\sim_R~y\subseteq Y \quad \equiv \quad \exists r \in
R. ~ \pi_X(r) = x \, \wedge \, \pi_Y(r) = y
\]
\end{proof}
%
In summary, whereas the category $\Rel$ is $\Span(\Set)$, powerset
relations are $\Span(\Rel) = \Span(\Span(\Set))$. We denote this
bicategory $\wp\Rel$.

\subsection{Compact-closed structure of $\wp\Rel$}




\section{Quantifiers as Relations}
\renewcommand{\wp}[1]{2^{#1}} It follows from what was said thus far
that equation \eqref{eq:q} exhibits a quantifier as a relation $\wp{Y}
\to \wp{Y}$.


So in order to interpret generailed quantifiers in $\Rel$, we look
into relations over powersets. 





%%% Local Variables: 
%%% mode: latex
%%% TeX-master: "Quant"
%%% End: 
